\documentclass[%
     %reprint,
%superscriptaddress,
%groupedaddress,
%unsortedaddress,
%runinaddress,
%frontmatterverbose, 
preprint,
%preprintnumbers,
%nofootinbib,
%nobibnotes,
%bibnotes,
amsmath,amssymb,
aps,
%pra,
%prb,
%rmp,
%prstab,
%prstper,
%floatfix,
]{revtex4-2}    
\usepackage{multirow}
\usepackage{graphicx}% Include figure files
\usepackage{dcolumn}% Align table columns on decimal point
\usepackage{bm}% bold math
%\usepackage{hyperref}% add hypertext capabilities
%\usepackage[mathlines]{lineno}% Enable numbering of text and display math
%\linenumbers\relax % Commence numbering lines

%\usepackage[showframe,%Uncomment any one of the following lines to test 
%%scale=0.7, marginratio={1:1, 2:3}, ignoreall,% default settings
%%text={7in,10in},centering,
%%margin=1.5in,
%%total={6.5in,8.75in}, top=1.2in, left=0.9in, includefoot,
%%height=10in,a5paper,hmargin={3cm,0.8in},
%]{geometry}
\usepackage[utf8x]{inputenc} % Включаем поддержку UTF8  
\usepackage[russian]{babel}  % Включаем пакет для поддержки русского языка 
\usepackage[normalem]{ulem}  % для зачекивания текста
\usepackage{indentfirst}

\usepackage[noend]{algorithmic}
\def\algorithmicrequire{\textbf{Вход:}}
\def\algorithmicensure{\textbf{Выход:}}
\def\algorithmicif{\textbf{если}}
\def\algorithmicthen{\textbf{то}}
\def\algorithmicelse{\textbf{иначе}}
\def\algorithmicelsif{\textbf{иначе если}}
\def\algorithmicfor{\textbf{для}}
\def\algorithmicforall{\textbf{для всех}}
\def\algorithmicdo{}
\def\algorithmicwhile{\textbf{пока}}
\def\algorithmicrepeat{\textbf{повторять}}
\def\algorithmicuntil{\textbf{пока}}
\def\algorithmicloop{\textbf{цикл}}
% переопределение стиля комментариев
\def\algorithmiccomment#1{\quad// {\sl #1}}

\usepackage{caption}
\usepackage{subcaption}
\usepackage{multirow}
\usepackage[table,xcdraw]{xcolor}
\begin{document}



\title{Сравнение антивирусного программного обеспечения}% Force line breaks with \\



\author{Васильев Михаил Владимирович}
\affiliation{%
 Студент 4 курса Московского Физико-Технического Института, физтех-школа “Радиотехника и компьютерные технологии” \\ Россия, г.Долгопрудный.\\
}%

\date{5 декабря 2022 г.}% It is always \today, today,
             %  but any date may be explicitly specified
             

%\begin{abstract}
%\begin{description}
%\item
%Проведён сравнительный анализ функциональных возможностей различного антивирусного программного обеспечения. Описана общепринятая методика сравнительного тестирования антивирусных продуктов. Проведено сопоставление общепринятой методики с результатами российских учёных.
%\end{description}
%\end{abstract}


%\keywords{Suggested keywords}%Use showkeys class option if keyword
                              %display desired
\maketitle

%\tableofcontents

\section{Аннотация}
Проведён сравнительный анализ функциональных возможностей различного антивирусного программного обеспечения. Описана общепринятая методика сравнительного тестирования антивирусных продуктов. Проведено сопоставление общепринятой методики с результатами российских учёных.

\section{Введение}
В настоящее время существует большое количество антивирусных программ и их методов тестирования, а следовательно, практически любая антивирусная программа (АП) может быть лучшей по результатам проверки. Ключевой проблемой конечного пользователя является выбор оптимального защитного ПО. Целью данного эссе является описание универсальных методов оценки АП, а также применение их для нахождения лучшего решения на рынке.


\section{Необходимость установки антивирусной программы}
$\textbf{Антивирусная программа}$ — специализированная программа для обнаружения компьютерных вирусов, а также нежелательных (считающихся вредорносными) программ и восстановления заражённых (модифицированных) такими программами файлов и профилактики — предотвращения заражения (модификации) файлов или операционной системы вредоносным кодом.

 Ключевой особенностью антивирусной защиты является то, что количество и качество вирусов меняется каждый день. Математически доказано, что ни одна технология не может защитить конечное устройство на 100\%. Пользователю необходимо самостоятельно подбирать продукты на рынке. Ситуация осложняется тем, что практически каждый производитель антивирусного ПО может заявлять, что его программа лучшая. Причиной данной проблемы является не обман потребителя, а множество методов проверки качества продукции. Несмотря на это приобретение оптимального защитного решения совершенно необходимо каждому пользователю персонального компьютера. \\


\section{Принцип действия АП}
Базируется в основном на 4 подходах к поиску вредоносных программ:

$\textbf{Метод поиска сигнатур}$ – основан на анализе существующих вирусов и выделении уникального битового кода (сигнатуры) характерного для определённой программы. После чего сигнатура заносится в специальную базу данных. Постоянное обновление этой базы данных позволяет антивирусным программам поддерживать свою эффективность. 

Ключевым преимуществом данного метода является малое количество ложных срабатываний, а недостатком – принципиальная невозможность нахождения новых угроз.

	$\textbf{Метод контроля целостности}$ – рассматривает неожиданное и беспричинное изменение данных на диске как сигнал подозрительной активности вредоносных программ. Факт нарушения целостности данных легко и быстро проверяется путём сравнения уже посчитанной контрольной суммы с настоящей. Если они не совпадают производится дополнительная проверка на совпадения вирусных сигнатур исполняемой программы и базы данных. Преимуществами данного метода являются: быстрота, так как проверка контрольных сумм занимает меньше времени, и возможность нахождения ранее неизвестных вирусов. 

	$\textbf{Метод сканирования подозрительных команд}$ – выявляет в программах признаки подозрительных команд и битовых последовательностей. После этого производятся дополнительные действия по проверке файла. Этот подход часто не способен выявлять новые угрозы.

$\textbf{Метод отслеживания поведения программ}$ – основан на анализе поведения запущенных программ. Для него необходимо активное присутствие человека, для подтверждения действий антивируса. Так как при наличии большого количества ложных срабатываний пользователь склонен к выбору оптимистичного сценария, часто используется режим эмулятора работы программы. Данный метод показал свою высокую эффективность в борьбе с известными и неизвестными вирусами.

$\textbf{Брандмауэр или сетевой экран}$ - предназначен для защиты от сетевых. Многие программы для соединения с удаленными компьютерами или серверами могут использовать небезопасные методы, оставляя уязвимости для проникновения извне. Суть работы брандмауэра в контроле как входящего, так и исходящего трафика путем ограничения возможности устанавливать соединения с определенными удаленными ресурсами. Самый наглядный метод защиты – белые и черные списки сетевых ресурсов. Черный список сетевых ресурсов – это список, например, сайтов, куда заходить нельзя, а белый список – это список ресурсов, куда только и можно заходить. Настройки брандмауэра позволяют обеспечить возможность сетевого взаимодействия только с проверенными ресурсами, отсекая все потенциально опасные и непроверенные. Недостаток брандмауэра вытекает из его достоинства: для качественной настройки файрвола требуются хорошие знания сетевых протоколов и особенностей работы сетевых приложений. Брандмауэр, работающий с настройками «по умолчанию» мало от чего способен защитить.

$\textbf{Почтовый Антивирус}$ - проверяет входящие и исходящие сообщения электронной почты на наличие в них вирусов и других программ, представляющих угрозу. Он запускается при старте, постоянно находится в оперативной памяти компьютера и проверяет все сообщения, получаемые или отправляемые.

$\textbf{Антиспам}$ — позволяет фильтровать почту от нежелательных писем. Зачастую подобные сообщения могут обладать вредоносными программами.

$\textbf{Родительский контроль}$ — программа в Интернете для предотвращения его предполагаемого негативного воздействия на ребёнка.

$\textbf{Резервное копирование}$ — процесс создания копии данных на носителе (жёстком диске, дискете и т. д.), предназначенном для восстановления данных в оригинальном или новом месте их расположения в случае их повреждения или разрушения.

$\textbf{VPN}$ (виртуальная частная сеть) — позволяет создать безопасное зашифрованное подключение между несколькими устройствами поверх уже работающей сети. Благодаря ей пользователи могут получить удаленный доступ к закрытым сетям, а также замаскировать свой трафик и действия в интернете.




\section{Решения на рынке}
$\textbf{Kaspersky Internet Security}$

Один из самых популярных программных продуктов, созданный лабораторией Касперского входящей в четвёрку ведущих мировых производителей программного обеспечения для защиты устройств. Предназначен прежде всего для пользователей операционной системы Windows. Существуют как базовые виды защиты, так и улучшенные решения (Kaspersky Internet security, Total Security). Основными преимуществами данного решения являются: антивирусные базы, которые регулярно обновляются, лёгкость в установке, отсутствие проблем с приобретением и оплатой подписки в связи с санкционной политикой запада. 

$\textbf{McAfee}$

Представляет собой антивирусное программное обеспечение американского разработчика McAfee Incorporated, которое работает с наиболее популярными операционными системами (Windows, Mac, Android). Плюсом является наличие бесплатной демо версии. Удобный интерфейс, достойный уровень защиты, малое время проверки на вирусы и обновление баз данных в автоматическом режиме выгодно отличают данное решение от других. 

$\textbf{Dr. Web Security Space}$

Создан российской компанией «Доктор Веб» и считается первой созданной антивирусной программой в мире. Этот продукт совместим с Windows и Android платформами. Отличительной особенностью является то, что данная программа для проведения анализа пользовательских данных не загружает их к себе на сервера, что говорит о ее хорошем уровне защиты. Автоматическое обновление баз данных вирусных носителей происходит каждый день. Сама программа работает на уровне ядра операционной системы.

$\textbf{Avira Antivir}$

Является антивирусной комплексной программой, созданной немецкой компанией Avira GmbH, которая более тридцати лет работает в сфере информационной безопасности. Продукт отлично совместим с основными операционными программами, прост в использовании и имеет бесплатную тестовую версию сроком действия до тридцати дней. Есть возможность настроить параметры под пользователя (отключение автозапуска, автоматическое или ручное управление). При обнаружении зараженного файла, система сразу его стирает с носителя инфорации без помещения в карантин.

$\textbf{Norton Security}$

Это разработка американской компании Symantec, которая
является лидером в создании программного обеспечения. Данный антивирус обладает
гибкой системой настроек пользователя под свои нужды (резервное копирование, оп-
тимизация пространства на диске, управление автозагрузкой), прост в использовании и
имеет хороший уровень технической поддержки. Сам интерфейс удобен и понятен для
пользователя.


\section{Анализ функциональных возможностей антивирусных программ}

\begin{table}[h]
\begin{tabular}{|l|l|l|l|l|l|}
\hline
Параметры                                                                & \begin{tabular}[c]{@{}l@{}}Kaspersky Internet \\ Security\end{tabular} & \begin{tabular}[c]{@{}l@{}}McAfee Total \\ Protection\end{tabular} & \begin{tabular}[c]{@{}l@{}}Dr. Web Security \\ Space\end{tabular} & \begin{tabular}[c]{@{}l@{}}Norton \\ Security\end{tabular} & \begin{tabular}[c]{@{}l@{}}Avira \\ Antivir\end{tabular} \\ \hline
\begin{tabular}[c]{@{}l@{}}Файловый \\ антивирус\end{tabular}            & Да                                                                     & Да                                                                 & Да                                                                & Да                                                         & Да                                                       \\ \hline
\begin{tabular}[c]{@{}l@{}}Почтовый \\ антивирус\end{tabular}            & Да                                                                     &                                                                    & Да                                                                & Да                                                         & Да                                                       \\ \hline
Сетевой экран                                                            & Да                                                                     & Да                                                                 & Да                                                                & Да                                                         & Да                                                       \\ \hline
\begin{tabular}[c]{@{}l@{}}Интернет \\ антивирус\end{tabular}            & Да                                                                     & Да                                                                 & Да                                                                & Да                                                         & Да                                                       \\ \hline
Анти-спам                                                                & Да                                                                     &                                                                    & Да                                                                & Да                                                         &                                                          \\ \hline
\begin{tabular}[c]{@{}l@{}}Родительский \\ контроль\end{tabular}         & Да                                                                     & Да                                                                 & Да                                                                & Да                                                         & Да                                                       \\ \hline
\begin{tabular}[c]{@{}l@{}}Резервное \\ копирование\end{tabular}         & Да                                                                     & Да                                                                 & Да                                                                & Да                                                         &                                                          \\ \hline
Менеджер паролей                                                         & Да                                                                     & Да                                                                 &                                                                   &                                                            & Да                                                       \\ \hline
Встроенный VPN                                                           &                                                                        &                                                                    &                                                                   &                                                            & Да                                                       \\ \hline
\begin{tabular}[c]{@{}l@{}}Наличие версии \\ для ОС Windows\end{tabular} & Да                                                                     & Да                                                                 & Да                                                                & Да                                                         & Да                                                       \\ \hline
\begin{tabular}[c]{@{}l@{}}Наличие версии \\ для ОС Android\end{tabular} & Да                                                                     & Да                                                                 & Да                                                                & Да                                                         & Да                                                       \\ \hline
\end{tabular}
\end{table}

\section{Общепринятая методика подбора антивирусного программного обеспечения}
Не смотря на обилие методов проверки АП, существует универсальный метод проверки антивирусного ПО. Он основан на деятельности некоммерческой организации WildList, которая сама тестированиями не занимается, а предоставляет тестерам базы данных диких вирусов. «Дикие вирусы» – это вирусы, свободно распространяющиеся по Всемирной сети и периодически атакующие компьютеры пользователей. Базы данных подобных вирусов обновляются ежемесячно и предоставляются тестирующим компаниям независимо.

$\textbf{AV-Comparatives}$ – один из мировых лидеров в области тестирования ПО для защиты от угроз. По результатам испытаний антивирусным продуктам выдается сертификат Real-World Protection.

В рамках сертификации продукты тестируются по категориям. Сегодня это 4 типа испытаний и, соответственно, 4 типа сертификатов:

• Advanced+ - испытываемый продукт должен распознать все вирусы из списка «In The Wild», выпущенного за два месяца до даты тестирования.

• Advanced - антивирус, получивший этот сертификат, должен не только обнаружить, но и вылечить систему от всех вирусов, найденных в Level 1.

• Standard – средний уровень защиты.

• Tested – тест не был пройден.

Таблица результатов:

% Please add the following required packages to your document preamble:
% \usepackage[table,xcdraw]{xcolor}
% If you use beamer only pass "xcolor=table" option, i.e. \documentclass[xcolor=table]{beamer}
\begin{table}[h]
\begin{tabular}{|l|l|l}
\cline{1-2}
Антивирус                                            & Оценка            &  \\ \cline{1-2}
{\color[HTML]{202122} Kaspersky   Internet Security} & Advanced +        &  \\ \cline{1-2}
McAfee Total Protection                              & Advanced +        &  \\ \cline{1-2}
Dr. Web Security Space                               & Нет данных        &  \\ \cline{1-2}
Norton Security                                      & Advanced + (2019) &  \\ \cline{1-2}
Avira Antivir                                        & Advanced +        &  \\ \cline{1-2}
\end{tabular}
\end{table}

По данным видно, что Kaspersky, McAfee и Aviva прошли испытания AV-Comparatives на высший балл. Тогда как Norton не проходил испытания в 2020 и 2021 годах. Dr.Veb в последнее время вообще в испытаниях участия не принимал.

\section{Методология проверки антивирусного ПО российскими учёными}
Проверка распространённых антивирусных программ проводится не только за рубежом, но и в России. Для этой цели комманда Anti-Malware.ru провела серию исследований и разработала перечень критериев по данной тематике. Далее описана методология проверки.

В каждом тестируемом антивирусе запускалась задача сканирования по требованию каталога с огромным количеством вирусных экземпляров. Тестовая база вирусов состояла из 64446 программ. Коллекция сформирована путем поиска в Интернете. Также все тестируемые программы на момент тестирования имели актуальные версии с обновленными базами данных.

Для проведения тестирования антивирусов на лечение активного заражения экспертной группой отбирались вредоносные программы по следующим критериям:

1. детектирование родительского файла всеми участвующими в тесте антивирусами;

2. способность маскировать свое присутствие;

3. способность противодействовать обнаружению со стороны антивируса;

4. способность восстанавливаться в случае удаления некоторых компонент;

5. распространенность и известность;

Отбор состоял из выбора наиболее сложных примеров, которые удовлетворяют всем приведённым выше примерам. Детектирование вирусов со стороны всех участвовавших в тести антивирусов было критически важным параметром для выбора вредоностных программ. Все используемые в тесте вредоносные программы были собраны экспертами во время распространения в Интернет (In The Wild). Каждый отобранный экземпляр вредоносной программы проверялся на работоспособность и установку на тестовой системе.

Следующая таблица приводит результаты тестрования наиболее популярных антивирусов:
% Please add the following required packages to your document preamble:
% \usepackage[table,xcdraw]{xcolor}
% If you use beamer only pass "xcolor=table" option, i.e. \documentclass[xcolor=table]{beamer}
\begin{table}[h]
\begin{tabular}{|l|l|l|l|l|}
\hline
Наименование                                       & Всего  & Вирусов & \% опознанных & Позиция в рейтинге \\ \hline
McAfee Total Protection                            & 65169  & 62605   & 96 & 1                  \\ \hline
Dr. Web Security Space                             & 86190  & 59621   & 69 & 2                  \\ \hline
Avira Antivir                                      & 76919  & 47552   & 62 & 3                  \\ \hline
{\color[HTML]{202122} Kaspersky Internet Security} & 109490 & 67205   & 61 & 4                  \\ \hline
Norton Security                                    & 84427  & 37015   & 44 & 5                  \\ \hline
\end{tabular}
\end{table}

\section{Вывод}
В данном эссе было представлено три различных независимых подхода к оценке антивирусных программ. Для конечного пользователя представляет интерес функциональный анализ и сертификация AV-Comparatives. По результатам данного исследования можно сказать, что при прочих равных выбор можно отдать McAfee.

\section{Источники}
https://cyberleninka.ru/article/n/razrabotka-metodiki-sravnitelnogo-testirovaniya-antivirusnyh-produktov/viewer

https://cyberleninka.ru/article/n/sravnitelnyy-analiz-antivirusnogo-programmnogo-obespecheniya/viewer

https://cyberleninka.ru/article/n/kompyuternye-virusy-i-antivirusy/viewer

https://www.av-comparatives.org/tests/real-world-protection-test-july-october-2022/

https://support.kaspersky.com/KESWin/10SP2/ru-RU/128014.htm


\end{document}